\documentclass[12pt]{article}

\usepackage{graphicx}			% Use this package to include images
\usepackage{amsmath}			% A library of many standard math expressions
\usepackage{amssymb}
\usepackage{amsfonts} % mathbb, etc.
\usepackage{esint} % closed integrals
\usepackage[margin=1in]{geometry}% Sets 1in margins. 
\usepackage{fancyhdr}			% Creates headers and footers
\usepackage{enumerate}          %These two package give custom labels to a list
\usepackage[shortlabels]{enumitem}
\usepackage{framed} % or, "mdframed"
\usepackage[framed]{ntheorem}

% Creates the header and footer. You can adjust the look and feel of these here.
\pagestyle{fancy}
\fancyhead[l]{Luca Mezzavilla}
\fancyhead[c]{MTH2010 Lecture \#1}
\fancyhead[r]{\today}
\fancyfoot[c]{\thepage}
\renewcommand{\headrulewidth}{0.2pt} %Creates a horizontal line underneath the header
\setlength{\headheight}{15pt} %Sets enough space for the header

\newframedtheorem{theorem}{Theorem}
\newframedtheorem{lemma}{Lemma}
\newframedtheorem{corollary}{Corollary}
\newframedtheorem{proposition}{Proposition}
\newframedtheorem{definition}{Definition}
\newframedtheorem{problem}{Problem}
\newframedtheorem{example}{Example}

\begin{document} %The writing for your homework should all come after this. 

\noindent You can modify the code in this template to make your first homework in LaTeX. You can put any \textit{initial text} in this section. 


%Enumerate starts a list of problems so you can put each homework problem after each item. 
\begin{enumerate}[start=1,label={\bfseries Question \arabic*:},leftmargin=1in] %You can change "Problem" to be whatever label you like. 
    \item This problem contains an equation in display mode. The definition of the Riemann Integral is 

    \[ \int_a^bf(x)dx=\lim_{||P||\rightarrow 0}\sum_{k=1}^nf(x_k^*)\Delta x_k\]

    Alternatively you can use $f(x)$ to write an equation in the middle of a sentence. 
    
    \item This is a problem that contains a table. I can reference Table \ref{tab:my_label} here. 


\begin{table}[h]
    \centering
    \begin{tabular}{|c|c|} \hline 
         Column 1& Column 2\\ \hline \hline 
         1& 2\\ \hline 
         3& 4\\ \hline
    \end{tabular}
    \caption{This is a caption}
    \label{tab:my_label}
\end{table}
  
     \item This is a problem about matrices. 

       \[ \begin{bmatrix}
        0 && 1\\
        1 && 0
    \end{bmatrix}\]

     \item This problem contains aligned equations. For $f(x)=x^2$,

     \begin{align*}
         f'(x)&=\lim_{h\to 0}\frac{f(x+h)-f(x)}{h}\\
         &= \lim_{h\to 0}\frac{(x+h)^2-x^2}{h}\\
         &= \lim_{h\to 0}\frac{2xh+h^2}{h}\\
         &= \lim_{h\to 0}2x+h\\
         &=2x
     \end{align*}
     

\newpage  %This creates a newpage

     
\item This \textbf{question} has \textit{multiple} parts
\begin{enumerate}
    \item This is part a
    \item This is part b
\end{enumerate}

\end{enumerate}

\begin{theorem}[Cayley-Hamilton Theorem]
  All square matrices satisfy their own characteristic polynomial.
\end{theorem}

\begin{corollary}[Example]
  A corollary
\end{corollary}

\begin{definition}[Orthonormal]
  Unit vectors which are linearly independent
\end{definition}

\[
  \oiint_S \vec{F} \cdot d\vec{S} = \iiint_V \nabla \cdot \vec{F} d\vec{V}
\]

\[
             
\]

\end{document}
